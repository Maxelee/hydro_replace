%% Beginning of file 'sample701.tex'
%%
%% Version 7.0.1. Created May 2025.
%% Version 7. Created January 2025.  
%%
%% AASTeX v7+ calls the following external packages:
%% times, hyperref, ifthen, hyphens, longtable, xcolor, 
%% bookmarks, array, rotating, ulem, and lineno 
%%
%% RevTeX is no longer used in AASTeX v7+.
%%
\documentclass[linenumbers,trackchanges]{aastex701}
\usepackage{amsmath}
\usepackage{bm}
%%
%% This initial command takes arguments that can be used to easily modify 
%% the output of the compiled manuscript. Any combination of arguments can be 
%% invoked like this:
%%
%% \documentclass[argument1,argument2,argument3,...]{aastex701}
%%
%% Six of the arguments are typestting options. They are:
%%
%%  twocolumn   : two text columns, 10 point font, single spaced article.
%%                This is the most compact and represent the final published
%%                derived PDF copy of the accepted manuscript from the publisher
%%  default     : one text column, 10 point font, single spaced (default).
%%  manuscript  : one text column, 12 point font, double spaced article.
%%  preprint    : one text column, 12 point font, single spaced article.  
%%  preprint2   : two text columns, 12 point font, single spaced article.
%%  modern      : a stylish, single text column, 12 point font, article with
%% 		  wider left and right margins. This uses the Daniel
%% 		  Foreman-Mackey and David Hogg design.
%%
%% Note that you can submit to the AAS Journals in any of these 6 styles.
%%
%% There are other optional arguments one can invoke to allow other stylistic
%% actions. The available options are:
%%
%%   astrosymb    : Loads Astrosymb font and define \astrocommands. 
%%   tighten      : Makes baselineskip slightly smaller, only works with 
%%                  the twocolumn substyle.
%%   times        : uses times font instead of the default.
%%   linenumbers  : turn on linenumbering. Note this is mandatory for AAS
%%                  Journal submissions and revisions.
%%   trackchanges : Shows added text in bold.
%%   longauthor   : Do not use the more compressed footnote style (default) for 
%%                  the author/collaboration/affiliations. Instead print all
%%                  affiliation information after each name. Creates a much 
%%                  longer author list but may be desirable for short 
%%                  author papers.
%% twocolappendix : make 2 column appendix.
%%   anonymous    : Do not show the authors, affiliations, acknowledgments,
%%                  and author contributions for dual anonymous review.
%%  resetfootnote : Reset footnotes to 1 in the body of the manuscript.
%%                  Useful when there are a lot of authors and affiliations
%%		    in the front matter.
%%   longbib      : Print article titles in the references. This option
%% 		    is mandatory for PSJ manuscripts.
%%
%% Since v6, AASTeX has included \hyperref support. While we have built in 
%% specific %% defaults into the classfile you can manually override them 
%% with the \hypersetup command. For example,
%%
%% \hypersetup{linkcolor=red,citecolor=green,filecolor=cyan,urlcolor=magenta}
%%
%% will change the color of the internal links to red, the links to the
%% bibliography to green, the file links to cyan, and the external links to
%% magenta. Additional information on \hyperref options can be found here:
%% https://www.tug.org/applications/hyperref/manual.html#x1-40003
%%
%% The "bookmarks" has been changed to "true" in hyperref
%% to improve the accessibility of the compiled pdf file.
%%
%% If you want to create your own macros, you can do so
%% using \newcommand. Your macros should appear before
%% the \begin{document} command.
%%
\newcommand{\vdag}{(v)^\dagger}
\newcommand\aastex{AAS\TeX}
\newcommand\latex{La\TeX}
%%%%%%%%%%%%%%%%%%%%%%%%%%%%%%%%%%%%%%%%%%%%%%%%%%%%%%%%%%%%%%%%%%%%%%%%%%%%%%%%
%%
%% The following section outlines numerous optional output that
%% can be displayed in the front matter or as running meta-data.
%%
%% Running header information. A short title on odd pages and 
%% short author list on even pages. Note that this
%% information may be modified in production.
%%\shorttitle{AASTeX v7.0.1 Sample article}
%%\shortauthors{The Terra Mater collaboration}
%%
%% Include dates for submitted, revised, and accepted.
%%\received{February 1, 2025}
%%\revised{March 1, 2025}
%%\accepted{\today}
%%
%% Indicate AAS Journal the manuscript was submitted to.
%%\submitjournal{PSJ}
%% Note that this command adds "Submitted to " the argument.
%%
%% You can add a light gray and diagonal water-mark to the first page 
%% with this command:
%% \watermark{text}
%% where "text", e.g. DRAFT, is the text to appear.  If the text is 
%% long you can control the water-mark size with:
%% \setwatermarkfontsize{dimension}
%% where dimension is any recognized LaTeX dimension, e.g. pt, in, etc.
%%%%%%%%%%%%%%%%%%%%%%%%%%%%%%%%%%%%%%%%%%%%%%%%%%%%%%%%%%%%%%%%%%%%%%%%%%%%%%%%
%%
%% Use this command to indicate a subdirectory where figures are located.
%%\graphicspath{{./}{figures/}}
%% This is the end of the preamble.  Indicate the beginning of the
%% manuscript itself with \begin{document}.

\begin{document}

\title{Responding to baryons with the baryonic response metric}

%% A significant change from AASTeX v6+ is in the author blocks. Now an email
%% address is required for each author. This means that each author requires
%% at least one of the following:
%%
%% \author
%% \affiliation
%% \email
%%
%% If these three commands are not available for each author, the latex
%% compiler will issue an error and if you force the latex compiler to continue,
%% it will generate an incomplete pdf.
%%
%% Multiple \affiliation commands are allowed and authors can also include
%% an optional \altaffiliation to indicate a status, i.e. Hubble Fellow. 
%% while affiliations are indexed as footnotes, altaffiliations are noted with
%% with a non-numeric footnote that is set away from the numeric \affiliation 
%% footnotes. NOTE that if an \altaffiliation command is used it must 
%% come BEFORE the \affiliation call, right after the \author command, in 
%% order to place the footnotes in the proper location. Because non-numeric
%% symbols are used, \altaffiliation should be used sparingly.
%%
%% In v7+ the \author command takes an optional argument which provides 
%% additional metadata about the author. Authors can provide the 16 digit 
%% ORCID, the surname (family or last) name, the given (first or fore-) name, 
%% and a name suffix, e.g. "Jr.". The syntax is:
%%
%% \author[orcid=0000-0002-9072-1121,gname=Gregory,sname=Schwarz]{Greg Schwarz}
%%
%% This name metadata in not shown, it is only for parsing by the peer review
%% system so authors can be more easily identified. This name information will
%% also be sent to the publisher so they can include it in the CROSSREF 
%% metadata. Including an orcid will hyperlink the author name to the 
%% author's ORCID page. Note that  during compilation, LaTeX will do some 
%% limited checking of the format of the ID to make sure it is valid. If 
%% the "orcid-ID.png" image file is  present or in the LaTeX pathway, the 
%% ORCID icon will appear next to the authors name.
%%
%% Even though emails are now required for each author, the \email does not
%% produce output in the compiled manuscript unless the optional "show" command
%% is used. For example,
%%
%% \email[show]{greg.schwarz@aas.org}
%%
%% All "shown" emails are show in the bottom left of the first page. Due to
%% space constraints, only a few emails should be shown. 
%%
%% To identify a corresponding author, use the \correspondingauthor command.
%% The command appends "Corresponding Author: " to the argument it appears at
%% the bottom left of the first page like the output from \email. 

\author[orcid=0000-0000-0000-0001,sname='North America']{Tundra North America}
\altaffiliation{Kitt Peak National Observatory}
\affiliation{University of Saskatchewan}
\email[show]{fakeemail1@google.com}  



%% Use the \collaboration command to identify collaborations. This command
%% takes an optional argument that is either a number or the word "all"
%% which tells the compiler how many of the authors above the command to
%% show. For example "\collaboration[all]{(DELVE Collaboration)}" wil include
%% all the authors above this command.
%%
%% Mark off the abstract in the ``abstract'' environment. 
\begin{abstract}

\end{abstract}


%% Keywords should appear after the \end{abstract} command. 
%% The AAS Journals now uses Unified Astronomy Thesaurus (UAT) concepts:
%% https://astrothesaurus.org
%% You will be asked to selected these concepts during the submission process
%% but this old "keyword" functionality is maintained in case authors want
%% to include these concepts in their preprints.
%%
%% You can use the \uat command to link your UAT concepts back its source.
\keywords{\uat{Galaxies}{573}}

%% From the front matter, we move on to the body of the paper.
%% Sections are demarcated by \section and \subsection, respectively.
%% Observe the use of the LaTeX \label
%% command after the \subsection to give a symbolic KEY to the
%% subsection for cross-referencing in a \ref command.
%% You can use LaTeX's \ref and \label commands to keep track of
%% cross-references to sections, equations, tables, and figures.
%% That way, if you change the order of any elements, LaTeX will
%% automatically renumber them.

\section{Introduction}
\label{sec:intro}

Realizing the full scientific potential of upcoming weak gravitational lensing surveys such as Euclid, the Vera C. Rubin Observatory's Legacy Survey of Space and Time (LSST), and the Nancy Grace Roman Space Telescope requires percent-level control of systematic uncertainties, including the impact of baryonic physics on the matter power spectrum and higher-order statistics. This level of control is made difficult with baryonic feedback from stellar winds, supernova explosions, active galactic nuclei (AGN) which redistribute matter within and around dark matter halos. This redistribution has the effect of suppressing the matter power spectrum by 10--20\% at scales $k \sim 1,h,\mathrm{Mpc}^{-1}$ and altering non-Gaussian statistics such as weak lensing peak counts and the matter bispectrum. If untreated, these effects can bias cosmological parameter constraints, particularly on $S_8 = \sigma_8(\Omega_m/0.3)^{0.5}$ by amounts exceeding statistical uncertainties from Stage-IV data.

Hydrodynamical simulations self-consistently model gas physics, star formation, and feedback providing the most accurate predictions of baryonic effects, but they remain computationally prohibitive for the large volumes and statistical ensembles required by modern surveys. A single flagship hydrodynamical run, such as IllustrisTNG or FLAMINGO, requires tens of millions of CPU hours and cannot easily be repeated across the high-dimensional parameter spaces needed for inference pipelines. 

This computational expense has driven the development of \emph{baryonic correction models} (BCMs), which attempt to capture baryonic effects by post-processing gravity-only dark matter simulations with analytic or semi-analytic prescriptions for gas ejection, stellar fractions, and concentration modifications. These models can be calibrated to match the matter power spectrum suppression observed in hydrodynamical simulations or inferred from external observables such as Sunyaev--Zel'dovich (SZ) and X-ray data making them powerful and physically motivated alternatives to hydrodynamical simulations.

Despite their computational efficiency, BCMs face a fundamental question of physical consistency: if a model is tuned to reproduce one statistic—for example, the 3D matter power spectrum $P(k)$—does it automatically reproduce the correct baryonic impact on other observables, such as the weak lensing convergence power spectrum $C_\ell$, peak counts, minima, or higher-order correlations? 

Recent evidence suggests the answer is no. Comparisons of BCMs against hydrodynamical benchmarks have revealed statistic-dependent discrepancies: a BCM may match $P(k)$ at the few-percent level while failing to reproduce peak count distributions or bispectra. More fundamentally, \citet{miller2025} demonstrated that even a direct \emph{halo-replacement} procedure—swapping DMO halos with their matched hydrodynamical counterparts out to $2R_{200}$ in IllustrisTNG—does not fully recover the hydrodynamical matter power spectrum. This discrepancy implies that some fraction of the baryonic effect arises from lower-mass halos, larger radii, or diffuse intergalactic components not captured by simple halo-based replacement. If even perfect halo profiles fail to reconstruct $P(k)$, then BCMs tuned to $P(k)$ must be compensating by placing baryonic modifications in regions of halo mass, radius, and redshift space that are inconsistent with hydrodymaical halos. It could then be true that these modifications then propagate inconsistently to other statistics.

This raises two critical questions. First, which halo masses, radii, and redshifts dominate the baryonic impact on a given observable? Understanding this attribution is essential for determining where BCMs must be most accurate and where approximations are tolerable. Second, how can we test whether a BCM is physically self-consistent across multiple observables? Matching one statistic is insufficient if the underlying density profiles of halos differ from hydrodynamical predictions in ways that matter for other statistics. Answering these questions requires a unified framework that quantifies the response of arbitrary observables to baryonic modifications in specific regions of halo parameter space and provides a diagnostic for cross-statistic consistency.

In this work, we introduce a self consistent formalism that addresses both questions as functions of mass radius and redshift. We define a family of hybrid ``Replace" density fields in which DMO halos at a redshift $z$ and above a mass threshold $M_{\min}$ are replaced by their hydrodynamical counterparts within a radial factor $\alpha R_{200}$, and construct a cumulative response function $F_S(M_{\min},\alpha)$ that measures the fraction of the total baryonic effect on observable $S$ captured by this replacement. By discretizing halo mass finite bins, we further define a tile response kernel $K_S(M_a,M_b;\alpha_i;z_k)$ that provides an empirical map of where baryonic effects originate in $(M,\alpha,z)$ space for each statistic. This kernel can be marginalized over mass or radius to answer questions such as which halo mass range contributes most to the convergence power spectrum at $\ell=1000$?'' or ``how far in radius does one need to model correctly for peak counts at signal-to-noise $\nu=4$?'' Crucially, the formalism applies to any summary statistic—matter power spectra, weak lensing two-point functions, peak counts, minima, PDFs, and higher-order moments—enabling direct comparisons of response patterns across observables.

We apply this framework to IllustrisTNG simulations, computing response kernels for the 3D matter power spectrum, projected 2D power spectra, weak lensing convergence power, and non-Gaussian peak statistics. We then perform baryon corrections with various BCMs to assess whether BCMs place baryonic effects in the correct halo parameter space. A BCM that matches $K_S^{\mathrm{Hydro}}$ for multiple statistics is physically consistent. 

This paper is organized as follows. In Section~\ref{sec:formalism}, we develop the cumulative and discrete response formalisms, define the tile kernel $K_S(M_a,M_b;\alpha_i;z_k)$, and establish the mathematical properties of response fractions. In Section~\ref{sec:simulations}, we describe the IllustrisTNG DMO and Hydro runs, the halo-matching procedure, and the construction of Replace density fields. Section~\ref{sec:results_matter} presents response kernels for matter power spectra, testing additivity and quantifying the contribution of different mass--radius shells. Section~\ref{sec:results_WL} extends the analysis to weak lensing observables, including convergence power and peak counts. Section~\ref{sec:BCM} compares BCM response kernels to hydrodynamical benchmarks and discusses implications for Stage-IV survey strategies. We summarize our findings and discuss future extensions in Section~\ref{sec:conclusions}.



\section{A Response Formalism}
\label{sec:formalism}
We begin with a general treatment of the response for some field level statistic to baryons. In general we can define a density field as 
\begin{equation}\label{eq:general_replace_density}
    \rho_{R}(\bm{x}, z) = \rho_D(\bm{x}, z)  + \sum_{i \in H} \left[\rho_H^i(\bm{x}, z) - \rho_{D}^i(\bm{x}, z)\right]
\end{equation}
where $\rho_D(\bm{x}, z)$ is the density in a dark matter only simulation at coordinat $\bm{x}$ and redshift $z$. Given some set of coordinates described by $H$, the dark matter density at that location, $\rho_D^i$ can be modified via a replacement with the hydro density, $\rho_H^i$. In the above, the framework is general such that $H$ can signify any chosen coordinates, but for this work, we will choose $H$ to be a set of halos with masses $M$ and radius factors $\alpha R_{200}$. To avoid double-counting where halo regions overlap, we assign priority by decreasing halo mass. For a given location $\bm{x}$, if multiple halos' replacement regions contain $\bm{x}$, only the most massive halo performs the replacement at that location. 
Mathematically, we define:

\begin{equation}
\rho_R(\bm{x},z) = \rho_D(\bm{x},z) + \sum_{i \in H(M,z)} \theta_i(\bm{x}) 
\left[\rho_H^i(\bm{x},\alpha) - \rho_D^i(\bm{x},\alpha)\right],
\end{equation}

where the indicator function is
\begin{equation}
\theta_i(\bm{x}) = \begin{cases}
1 & \text{if } |\bm{x}-\bm{x}_i| < \alpha R_{200}^i \text{ and } i = \arg\max_j M_j 
\text{ among all } j \text{ with } |\bm{x}-\bm{x}_j| < \alpha R_{200}^j, \\
0 & \text{otherwise}.
\end{cases}
\end{equation}
This ensures that each spatial point receives replacement from at most one halo.

To make the intuition clearer, we will rewrite Eq.~\ref{eq:general_replace_density} as 
\begin{equation}\label{eq:Mass_alpha_z_replace_density}
    \rho_{R}(\bm{x}, z) = \rho_D(\bm{x}, z)  + \sum_{i \in H(M, z)} \left[\rho_H^i(\bm{x}, \alpha) - \rho_{D}^i(\bm{x}, \alpha)\right],
\end{equation}
to make clear that we iterate over some set of halos with masses $M$ and at redshift $z$, and for each, we replace within regions of $\alpha R_200$. 

As an example, consider the case $M_{\min} = 10^{13}\,M_\odot h^{-1}$ and $\alpha_0=1$. 
For each halo $i$ satisfying these criteria, we define extraction regions:
\begin{equation}
\rho_{D}^{i}(\bm{x};\alpha=1) =
\begin{cases}
\rho_{D}(\bm{x}) & \text{if } |\bm{x}-\bm{x}_i| < R_{200}^{i} \text{ AND } M^i \geq 10^{13}\,M_\odot h^{-1},\\
0 & \text{otherwise},
\end{cases}
\end{equation}

\begin{equation}
\rho_{H}^{i}(\bm{x};\alpha=1) =
\begin{cases}
\rho_{H}(\bm{x}) & \text{if } |\bm{x}-\bm{x}_i| < R_{200}^{i} \text{ AND } M^i \geq 10^{13}\,M_\odot h^{-1},\\
0 & \text{otherwise}.
\end{cases}
\end{equation}

In the limit of $M_{\rm min}$ goes to 0 or $\alpha$ becomes $\infty$, the field becomes purely hydro, and inversely, the field becomes purely dark matter only. It can then be thought of that $\rho_R$ interpolates between the DMO and hydro fields as the mass threshold and radius factor are modified. 

For any statistic that we measure from the resulting replace field ($S_R$), we can assume that its value is somewhere between the dark matter only ($S_D$) and hydro versions ($S_H$) of the same statistic allowing us to define some response fraction, 
\begin{equation}\label{eq:cum_resp_func}
    F_S(M_{\rm min}, \alpha) = \dfrac{S_R(M_{\rm min}, \alpha) - S_D}{S_H-S_D}.
\end{equation}
For a fixed $\alpha$ and redshift $z$, we can see that the fraction approaches $0$ as the minimum mass approaches infinity, and 1 as it approaches 0. Or intuitively, when no mass is replaced, the fraction is 0 reflecting that the statistic extracted from the replace field is the same as in a dark matter only simulation, while when all masses are included, the field and resulting statistic is purely hydro. 


The above definitions using $M_{\rm min}$ are inherently cumulative as they incorporate all halos above some threshold mass, and replacement regions in spheres according to $\alpha$. The response fraction then tells us for a given statistic, how low, and how far out do we need to go to capture the full baryonic effects on the given statistic. For example, peak counts, which are sensitive to the collapsed cores of massive halos may show that one must only match density profiles of massive $10^{13}$ and above halos out to $0.5 \times R200$, while the power spectrum may require more. We will show in \S\ref{sec:} various statistics and how much BCMs must do to approach them. 

In the discrete version, we partition halo masses and radii into finite bins 
and define the response from a specific mass-radius tile. For a mass bin 
$[M_a, M_{a+1})$ and radius shell $[\alpha_i R_{200}, \alpha_{i+1} R_{200})$, 
we compute a Replace field with halos in that bin replaced only within the 
specified radius shell. The tile response fraction is then:
\begin{equation}\label{eq:disc_resp_func}
    \Delta F_S(M_a, M_{a+1}; \alpha_i, \alpha_{i+1}) = 
    \dfrac{S_R(M_a, M_{a+1}; \alpha_i, \alpha_{i+1}) - S_D}{S_H-S_D},
\end{equation}
where $S_R(M_a, M_{a+1}; \alpha_i, \alpha_{i+1})$ is the statistic measured from 
the Replace field with halos in the specified mass-radius bin activated. Under the assumption that contributions from different mass-radius tiles combine 
additively, we can recover the cumulative response as:
\begin{equation}\label{eq:additive_approx}
    F_S(M_{\min}, \alpha) \approx \sum_{\substack{a: M_a \geq M_{\min} \\ 
    i: \alpha_i \leq \alpha}} 
    \Delta F_S(M_a, M_{a+1}; \alpha_i, \alpha_{i+1}),
\end{equation}
where the sum extends over all mass bin indices $a$ with $M_a \geq M_{\min}$ 
and all radius bin indices $i$ with $\alpha_i \leq \alpha$. Deviations from 
this relation quantify the nonlinearity of the map $S[\rho]$. To diagnose the degree of nonlinearity, we define the fractional non-additivity as:
\begin{equation}\label{eq:non_additivity}
    \epsilon_S(M_{\min}, \alpha) = \dfrac{F_S(M_{\min}, \alpha) -
    \sum_{\substack{a: M_a \geq M_{\min} \\ i: \alpha_i \leq \alpha}} 
    \Delta F_S(M_a, M_{a+1}; \alpha_i, \alpha_{i+1})}{F_S(M_{\min}, \alpha)}.
\end{equation}
Values $|\epsilon_S| \ll 1$ indicate that mass-radius bins contribute approximately 
additively to the observable $S$, while larger values signal significant nonlinear 
interactions between different regions of halo parameter space.

With the above formalism we can now explore how replacing halos from dark matter only simulations with hydro particles effects various statistics. We will approach this systematically and increasing in complexity. First we will compute the density fields of hydro, dmo and replaced for various redshifts. We will then compute 3D power spectra for each of these fields. We will perform the cumulative fractional response analysis to learn the relationship between masses, radius factors and the power spectrum and observe how this compares to the differences in the density profiles between the hydro and dmo. We will then perform replacements in mass bins at fixed radius factors to compute the differential response factor. We will use the replaced density fields to generate weak lensing maps with ray tracing. We will perform the same cumulative and differential fractional response analyses on various weak lensing statistics. Finally we will compute baryon corrections following the parameterizations of Arico, Schneider19 and Schneider25 on the illustrisTNG dark simulation. We will compare all of the statistics previously investigated with these BCM's to compare how effective these bcms are. 

\section{Simulations and Data Generation}

\subsection{The IllustrisTNG Suite}
We use the IllustrisTNG-300 simulation, though in practice this formalism could work for any simulation that contains matched initial conditions between a dmo and hydro simulation. 

\subsection{Halo Catalogs and Matching}
We use the FoF/Subfind catalogs to find matched halos between the Nbody simulation and hydro simulation. For this we perform a bijective matching and set a tolerance of 0.7 

\subsection{Ray-Tracing Pipeline}
We use the same procedure as in $\kappa$TNG, utilizing the multi-plane algorithm. 

\section{The Baryonic Response of the Matter Field}

\subsection{3D Matter Power Spectrum}
\begin{figure}
    \centering
    \includegraphics[width=\linewidth]{imgs/suppression_and_fraction.pdf}
    \caption{Caption}
    \label{fig:suppression_and_fraction}
\end{figure}
In Fig.~\ref{fig:suppression_and_fraction}, we show the power spectrum suppression between dmo and hydro for various $M_min$ values and with a constant radius factor of $\alpha=5$. We can see clearly that none of the models perfectly match the power spectrum. In the bottom panel, we compute the cumulative fractional response for each of the mass replacements which highlights the influence of adding certain mass bins. For galaxy clusters above $10^13.5$, properly matching them out to 5$R_200$ accounts for nearly half of the power spectrum suppression. On the other hand, including galaxies betweem $10^12$ and $10^12.5$ hardly improves the fractional response, indicating, that the power spectrum information is saturated before these masses, and that the rest of the suppression is likely outside of even these lower mass halos. 
\subsection{Cumulative Response}

\begin{figure}
    \centering
    \includegraphics[width=\linewidth]{imgs/fig02_cumulative_response.pdf}
    \caption{Caption}
    \label{fig:cum_response}
\end{figure}
In Fig.~\ref{fig:cum_response}, we show the mean response of the power spectrum between scales of $k=1-30$. We show in the left pannel the effect as a function of mass, where each line is at a fixed radius factor, where on the right panel we show the inverse, which is the fractional response as a function of radius factor, colored by halo mass. 


\subsection{Test of Additivity}


\section{The Baryonic Response of Weak Lensing Observables}
\subsection{Convergence Power Spectrum}

\begin{figure}
    \centering
    \includegraphics[width=\linewidth]{imgs/F_cl.pdf}
    \caption{Caption}
    \label{fig:placeholder}
\end{figure}
\subsection{PDF, Peak Counts, Minima}
\subsection{Minkowski Functionals and Wavelets}
\subsection{Statistic-Dependent Response}

\section{Benchmarking Baryonic Correction Models}
\subsection{BCM Implementations}

\subsection{Profile Comparisons}

\begin{figure}
    \centering
    \includegraphics[width=\linewidth]{imgs/bcm_vs_hydro_profiles.pdf}
    \caption{Caption}
    \label{fig:placeholder}
\end{figure}

\subsection{Consistency Check}
\section{Discussion}
\subsection{Physical Interpretation}
\subsection{Implications for Stage-IV Surveys}
\subsection{Limitations}

\section{Conclusion}







\end{document}
